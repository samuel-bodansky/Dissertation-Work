% $Header: /cvsroot/latex-beamer/latex-beamer/solutions/conference-talks/conference-ornate-20min.en.tex,v 1.7 2007/01/28 20:48:23 tantau Exp $

\documentclass[10pt]{beamer}



\mode<presentation>
{
  \usetheme{Warsaw}
  % or ...

  \setbeamercovered{transparent}
  % or whatever (possibly just delete it)
}


\usepackage[english]{babel}
% or whatever

\usepackage[latin1]{inputenc}
% or whatever

\usepackage{times}
\usepackage{graphicx}

\usepackage{tikz}
\usepackage{lineno}
\usepackage[english]{babel}
\usepackage{seqsplit}
\usepackage[nottoc]{tocbibind} 
\usepackage[utf8]{inputenc}
\usepackage{amsmath}
\usepackage{amssymb}
\usepackage{amsthm}
% \newtheorem{theorem}{Theorem}
% \newtheorem{lemma}[theorem]{Lemma}
% \newtheorem{corollary}{Corollary}[theorem]
% \newtheorem{definition}{Definition}
% \newtheorem{proposition}{Proposition}
% \newtheorem{example}{Example}

\title[Part C Dissertation] % (optional, use only with long paper titles)
{Extensions of number fields with small ramification}

\subtitle
{}

\author % (optional, use only with lots of authors)
{Samuel Bodansky}


\institute % (optional, but mostly needed)
{
  University of Oxford
  }
.

\date % (optional, should be abbreviation of conference name)
{February 2019}


\begin{document}

\begin{frame}
  \titlepage
\end{frame}


\section{Goal}
\subsection{Goal}
%*********************************************************
\begin{frame}{Aim of Project}
Goal: For a given number field K, what is the maximal unramified extension $K^{ur}$ of K is and also what is the structure of $G_K^{ur}$ is.

\end{frame}

\begin{frame}{Notation}
Here is a summary of relevant notation for this talk. 
\end{frame}
\section{Class Field Theory}
\begin{frame}{CFT Definitions}
    \begin{definition}
    A prime ideal $\mathfrak{p}$ in $\mathcal{O}_K$ factors in an extension L of K as
\par
$\mathfrak{p}\mathcal{O}_L=\mathfrak{B}_1^{e_1}...\mathfrak{B}_m^{e_m}$, where $\mathfrak{B}_i$ are ideals in $\mathcal{O}_L$ intersecting $\mathcal{O}_K$ at $\mathfrak{p}$. Each $e_i\geq1$ and if $e_i>1$ for some i then we say that $\mathfrak{p}$ ramifies in L. If $e_i=1$ for all i then we say $\mathfrak{p}$ splits in L.
    \end{definition}
    \begin{definition}
    Let H be a subgroup of the class group C of K. A finite unramified abelian extension L of K is said to be a $\textit{class field}$ for H if the prime ideals of K splitting in L are exactly those in $\tilde{H}$.
    \end{definition}
    
\end{frame}
\begin{frame}{Key Result about Unramified Extensions}
The following theorem will allow us to define $K^{ur}$:
    \begin{theorem}
The compositum of two finite unramified extensions of K is also unramified, and so the union $K^{ur}$ of all unramified extensions is also an unramified extension of K. The residue field $\tilde{k}$ of $K^{ur}$ is an algebraic closure of the residue field k of K.
\end{theorem}

\end{frame}
\begin{frame}{Proof of Theorem}
    \begin{proof}
    Suppose K is a number field and $L,\tilde{L}$ are extensions of K. Let $\mathfrak{p}$ be an ideal unramified in $L$ and $\tilde{L}$. Let $P$ be a prime lying over $\mathfrak{p}$ in $L\tilde{L}$. Let $M$ be the minimal normal extension of $L\tilde{L}$, i.e. the normal closure. Suppose $Q$ is a prime in $M$ lying over $P$, then we have 
    \begin{equation}
        \mathfrak{p}\subseteq P \subseteq Q
    \end{equation}
    Let $E=E(Q|\mathfrak{p})$ denote the inertia group of $Q$ at $\mathfrak{p}$. Since $\mathfrak{p}$ is unramified in $L$ and $\tilde{L}$, it follows that $Q \cap \mathcal{O}_L$ and $Q \cap \mathcal{O}_\tilde{L}$ are unramified in $L$ and $\tilde{L}$ respectively. But the inertia field $M^E$ is the largest field in which $\mathfrak{p}$ is unramified, so it follows that $L\subseteq M^E$ and $\tilde{L} \subseteq M^E $, whence $L\tilde{L}\subseteq M^E$ and $\mathfrak{p}$ is unramified in $L\tilde{L}$.
\end{proof}

\end{frame}
\section{$\mathbb{Q}$,P-adics and Cyclotomics}
\subsection{Examples in $\mathbb{Q}$}
\begin{frame}{Extensions of $\mathbb{Q}$}
    \begin{theorem}
     Suppose that K be a field extension of $\mathbb{Q}$ with discriminant D. Then a prime p
ramifies in K if and only if $p|D$.   
    \end{theorem}
    \begin{corollary}
A field extension K of $\mathbb{Q}$ ramifies at precisely one prime iff the absolute value of disc(K) is a prime power.
\end{corollary}
\begin{example}
    Suppose $f(x)=x^5-5x^4-2x^3+x^2-3x+1$, and K is the splitting field of f over $\mathbb{Q}$. Then $disc(K)=-3442951=151^3$ and so 151 is the only prime which ramifies in K over $\mathbb{Q}$. Note that f(x) is an irreducible quintic with precisely three real roots, meaning that $\Gamma(K/\mathbb{Q})\cong\ S_5$ and K is an unsolvable extension of $\mathbb{Q}$.
\end{example}
\end{frame}
\subsection{P-adics and Cyclotomics}
\begin{frame}{P-adics results}
\begin{block}{P-adics}
Suppose $\zeta$ is an $n_{th}$ root of unity. Let K be a number field, $L=K(\zeta)$. Let $\mathcal{O}_L,\mathcal{O}_K$ be their respective valuation rings and $\lambda,\kappa$ be their respective residue class fields. \begin{proposition}
L is a degree n unramified extension of K.
\end{proposition}

Suppose now that $\zeta$ is a primitive n-th root of unity, with $n=kp^{m}$ where k is prime to p. We have the following inclusions:
\end{block}
\begin{example}[Field Inclusions]
      $\mathbb{Q}_{p}=K\subseteq K^{tur}=K(\zeta_k)\subseteq K^{ur} = K^{tur}(\zeta_{p})\subseteq K(\zeta_n)$
\end{example}
\end{frame}
\begin{frame}{Maximal Unramified Extension of P-adic}
    We first use the following lemma 
    \begin{lemma}
Suppose p is a prime and gcd(n,p)=1. Then there exists $m\in\mathbb{N}$ such that $n|p^m-1$.
\end{lemma}
We then have the following:  \begin{theorem}
    Let $K=\mathbb{Q}_p$. Then $K^{ur}$ is K adjoined with all roots of unity with order prime to p. 
\end{theorem}

\end{frame}
\begin{frame}{Cyclotomics Results}
    \begin{theorem}
    Let $\zeta$ denote the $p^m-th$ root of unity, 
    Then we have the following results:
    \begin{enumerate}
        \item $[\mathbb{Q}_{p}(\zeta):\mathbb{Q}_p]$ is totally ramified with degree $(p-1)p^{m-1}$
        \item $\Gamma(\mathbb{Q}_{p}(\zeta):\mathbb{Q}_p)\cong (\mathbb{Z}/p^{m}\mathbb{Z})^{*}$
        \item $\mathbb{Z}_{p}(\zeta)$ is the valuation ring of $\mathbb{Q}_{p}(\zeta)$
        \item $1-\zeta$ is a prime element of $\mathbb{Z}_{p}(\zeta)$ with $N(1-\zeta)=p$
    \end{enumerate}
\end{theorem}
\end{frame}
\section{Hilbert Class Field}
\begin{frame}{Definition of Hilbert Class Field}
    \begin{definition}
The \textbf{class field} of the trivial subgroup of C(K) is called the Hilbert class field of K. 
\end{definition}
\begin{enumerate}
    \item It is maximal abelian extension, L of K unramified at all primes of K. 
    \item  The degree of its extension over K is equal to the class number of K, 
    \item Its Galois group is isomorphic to the ideal class group of K, taking Frobenius elements as prime ideals of K.
    \item If K is a unique factorisation domain then K is equal to its own Hilbert class field. 
\end{enumerate}
\end{frame}
\begin{frame}{Examples of Hilbert Class Fields}
    \begin{example}
    Suppose K=$\mathbb{Q}$. Then K is its own Hilbert class field. 
\end{example}
\begin{example}
    Suppose $K=\mathbb{Q}[\sqrt{-5}]$, and take $L=\mathbb{Q}[\sqrt{-1},\sqrt{-5}]$. L is an unramified degree 2 extension of K and so the Hilbert class field of K is $K[\sqrt{-1}]$.
\end{example}
\begin{example}[Hasse]
    Suppose $K=\mathbb{Q}[\sqrt{-31}]$ with class number 3. Let L be its Hilbert class field. Then L is defined by a root of the polynomial \begin{equation}
        x^3+\frac{3+\sqrt{-31}}{2}x^2+\frac{-3+\sqrt{-31}}{2}x -1 =0.
    \end{equation}
\end{example}
\end{frame}
\begin{frame}{Hilbert Class Field Applications Theorem}
    \begin{theorem}
Theorem: Let L be the Hilbert class field of $K=\mathbb{Q}(\sqrt{-n})$, where n is squarefree, $n\not\equiv 3(4)$, so that
\begin{equation}
    \mathcal{O}_K=\mathbb{Z}(\sqrt{-n})
\end{equation}
Assume that p is an odd prime not dividing n, and that x,y are integers. Then
$p=x^2+ny^2 \Longleftrightarrow$ p splits completely in L.
\end{theorem}
\end{frame}
\begin{frame}{Primes of the form $x^2+ny^2$}
    \begin{corollary}
Suppose $K=\mathbb{Q}{\sqrt{-n}}$ n is positive,squarefree, $n\not\equiv 3(4)$, so that $d_K=-4n$. If p is an odd prime not dividing n, then
\begin{equation} p=x^2+ny^2 \Longleftrightarrow p\:splits\:completely\:in\:the\:Hilbert\:class\:field\:of\:K.
\end{equation}
Hence the primes of the form $x^2+ny^2$ characterise the Hilbert Class field of $\mathbb{Q}(\sqrt{-n})$.
\end{corollary}
\end{frame}
\end{document}
