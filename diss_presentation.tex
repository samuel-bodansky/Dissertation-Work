% $Header: /cvsroot/latex-beamer/latex-beamer/solutions/conference-talks/conference-ornate-20min.en.tex,v 1.7 2007/01/28 20:48:23 tantau Exp $

\documentclass[10pt]{beamer}

% This file is a solution template for:

% - Talk at a conference/colloquium.
% - Talk length is about 20min.
% - Style is ornate.



% Copyright 2004 by Till Tantau <tantau@users.sourceforge.net>.
%
% In principle, this file can be redistributed and/or modified under
% the terms of the GNU Public License, version 2.
%
% However, this file is supposed to be a template to be modified
% for your own needs. For this reason, if you use this file as a
% template and not specifically distribute it as part of a another
% package/program, I grant the extra permission to freely copy and
% modify this file as you see fit and even to delete this copyright
% notice. 


\mode<presentation>
{
  \usetheme{Warsaw}
  % or ...

  \setbeamercovered{transparent}
  % or whatever (possibly just delete it)
}


\usepackage[english]{babel}
% or whatever

\usepackage[latin1]{inputenc}
% or whatever

\usepackage{times}
\usepackage{graphicx}

\usepackage{tikz}
\usepackage{lineno}
\usepackage[english]{babel}
\usepackage{seqsplit}
\usepackage[nottoc]{tocbibind} 
\usepackage[utf8]{inputenc}
%\usepackage[T1]{fontenc}
% Or whatever. Note that the encoding and the font should match. If T1
% does not look nice, try deleting the line with the fontenc.


\title[Part C Dissertation] % (optional, use only with long paper titles)
{Extensions of number fields with small ramification}

\subtitle
{}

\author % (optional, use only with lots of authors)
{Samuel Bodansky}
% - Give the names in the same order as the appear in the paper.
% - Use the \inst{?} command only if the authors have different
%   affiliation.

\institute % (optional, but mostly needed)
{
  University of Oxford
  }
% - Use the \inst command only if there are several affiliations.
% - Keep it simple, no one is interested in your street address.

\date % (optional, should be abbreviation of conference name)
{12 July, 2010}
% - Either use conference name or its abbreviation.
% - Not really informative to the audience, more for people (including
%   yourself) who are reading the slides online

%\subject{Theoretical Computer Science}
% This is only inserted into the PDF information catalog. Can be left
% out. 



% If you have a file called "university-logo-filename.xxx", where xxx
% is a graphic format that can be processed by latex or pdflatex,
% resp., then you can add a logo as follows:

%\pgfdeclareimage[height=1.5cm]{university-logo}{ox_man_logo}
%\logo{\pgfuseimage{university-logo}}



% Delete this, if you do not want the table of contents to pop up at
% the beginning of each subsection:
%\AtBeginSubsection[]
%{
%  \begin{frame}<beamer>{Outline}
%    \tableofcontents[currentsection,currentsubsection]
%  \end{frame}
%}


% If you wish to uncover everything in a step-wise fashion, uncomment
% the following command: 

%\beamerdefaultoverlayspecification{<+->}


\usepackage{amsmath}
\usepackage{amssymb}
\usepackage{amsthm}
% \newtheorem{theorem}{Theorem}
% \newtheorem{lemma}[theorem]{Lemma}
% \newtheorem{corollary}{Corollary}[theorem]
% \newtheorem{definition}{Definition}
% \newtheorem{proposition}{Proposition}
% \newtheorem{example}{Example}

\begin{document}

\begin{frame}
  \titlepage
\end{frame}

%\begin{frame}{Outline}
%  \tableofcontents
  % You might wish to add the option [pausesections]
%\end{frame}


% Structuring a talk is a difficult task and the following structure
% may not be suitable. Here are some rules that apply for this
% solution: 

% - Exactly two or three sections (other than the summary).
% - At *most* three subsections per section.
% - Talk about 30s to 2min per frame. So there should be between about
%   15 and 30 frames, all told.

% - A conference audience is likely to know very little of what you
%   are going to talk about. So *simplify*!
% - In a 20min talk, getting the main ideas across is hard
%   enough. Leave out details, even if it means being less precise than
%   you think necessary.
% - If you omit details that are vital to the proof/implementation,
%   just say so once. Everybody will be happy with that.


%*********************************************************
%*********************************************************
%*********************************************************
\section{Goal}
\subsection{Goal}
%*********************************************************
\begin{frame}{Aim of Project}
Goal: For a given number field K, what is the maximal unramified extension $K^{ur}$ of K is and also what is the structure of $G_K^{ur}$ is.

\end{frame}

\begin{frame}{Notation}
Here is a summary of relevant notation for this talk. 
\end{frame}
%*********************************************************  
\section{Initial Examples}
%*********************************************************  
\begin{frame}{Extensions of $\mathbb{Q}$}
    \begin{theorem}
     Suppose that K be a field extension of $\mathbb{Q}$ with discriminant D. Then a prime p
ramifies in K if and only if $p|D$.   
    \end{theorem}
    \begin{corollary}
A field extension K of $\mathbb{Q}$ ramifies at precisely one prime iff the absolute value of disc(K) is a prime power.
\end{corollary}
\begin{example}
    Suppose $f(x)=x^5-5x^4-2x^3+x^2-3x+1$, and K is the splitting field of f over $\mathbb{Q}$. Then $disc(K)=-3442951=151^3$ and so 151 is the only prime which ramifies in K over $\mathbb{Q}$. Note that f(x) is an irreducible quintic with precisely three real roots, meaning that $\Gamma(K/\mathbb{Q})\cong\ S_5$ and K is an unsolvable extension of $\mathbb{Q}$.
\end{example}
\end{frame}
\subsection{Hilbert Class Field}
\begin{frame}{Definition of Hilbert Class Field}
    \begin{definition}
The \textbf{class field} of the trivial subgroup of C(K) is called the Hilbert class field of K. 
\end{definition}
\begin{enumerate}
    \item It is maximal abelian extension, L of K unramified at all primes of K. 
    \item  The degree of its extension over K is equal to the class number of K, 
    \item Its Galois group is isomorphic to the ideal class group of K, taking Frobenius elements as prime ideals of K.
    \item If K is a unique factorisation domain then K is equal to its own Hilbert class field. 
\end{enumerate}
\end{frame}
\begin{frame}{Examples}
    \begin{example}
    Suppose K=$\mathbb{Q}$. Then K is its own Hilbert class field. 
\end{example}
\begin{example}
    Suppose $K=\mathbb{Q}[\sqrt{-5}]$, and take $L=\mathbb{Q}[\sqrt{-1},\sqrt{-5}]$. L is an unramified degree 2 extension of K and so the Hilbert class field of K is $K[\sqrt{-1}]$.
\end{example}
\end{frame}

\end{document}
